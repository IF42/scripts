\chap Zachytávání chyb
{\bf Chyba} v softwaru je nesprávné nebo neočekávané chování programu nebo systému, které se vyskytuje při jeho spuštění, používání nebo vývoji. Chyby mohou mít různé příčiny, od problémů v kódu až po nesprávné požadavky od uživatele. Mohou být objeveny během vývoje, testování nebo při ostrém používání aplikace.

Chyby se obvykle dělí do několika kategorií:

\begitems\style n
* {\bf Kompilační chyby} - chyby, které vznikly chybou ve zdrojových kódech, nebo chybným nastavením překladu
* {\bf Logické chyby} - Chyby, kdy program běží bez výjimek, ale neprovádí správně požadovanou funkci, což vede k nesprávným výsledkům.
* {\bf Chyby běhu} - Chyby, které se objeví během vykonávání programu, například dělení nulou, přístup k neexistujícímu souboru nebo pokus o přístup k nealokované paměti.
* {\bf Chyby návrhu} - Chyby, které vznikají na úrovni architektury nebo návrhu softwarového systému a mohou mít dlouhodobý vliv na jeho funkčnost a údržbu.
* {\bf Chyby uživatele} - Chyby způsobené nesprávným použitím programu nebo nedorozuměním mezi uživatelem a systémem.
\enditems

Chyby, které mohou nastat ve fázi kompilace, se pak dělí na:
\begitems\style n
* {\bf Lexikální chyby} - Chyby vzniklé na úrovni slovní reprezentace zdrojového kódu, například chyba v zápisu desetinného čísla, identifikátoru nebo klíčového slova
* {\bf Syntaktické chyby} - Chyby, které vznikají při porušení pravidel syntaxe programovacího jazyka. Tyto chyby jsou obvykle odhaleny během fáze kompilace nebo interpretace kódu.
* {\bf Sémantické chyby} - Chyby, které vznikají ve chvíly kdy je zdrojový kód syntakticky správny, ale kód přesto není platný, například nesprávné použití operací na datový typ, nebo změna hodnoty konstantní proměnné
* {\bf Logické chyby} - Chyby, které nastanou ve chyvíly kdy je kód ve všech směrech platný, ale přesto jeho použití nedává smysl, například nekonečné smyčky, mrtvý kód, ...
\enditems

{\bf Varování} v softwaru jsou upozornění na potenciální problémy nebo neoptimální chování programu, které nezpůsobují okamžité selhání nebo nesprávné výsledky, ale mohou vést k chybám nebo neefektivnímu fungování v budoucnu. Na rozdíl od chyb, které obvykle zastavují běh programu nebo způsobují nesprávné chování, varování upozorňují vývojáře nebo uživatele na situace, které by si zasloužily pozornost, ale nebrání v pokračování programu.

\sec Zachytávání chyb na úrovni kompilátoru
Úlohou zachytávání chyb je detekovat chyby, které mohou nastat v různých fázích překladu a předat tuto informaci uživatelovi, který tak může podniknout příslušné kroky k nápravě. Vztah mezi chybami a varováními spočívá v tom, jakým způsobem kompilátor reaguje na různé typy problémů, které se vyskytují v kódu během procesu překladu. Tyto dvě kategorie odrážejí rozdílné úrovně závažnosti problémů a určují, jakým způsobem kompilátor zachází s jejich detekováním a hlášením.

\sec Metody zachytávání chyb
Kompilátory používají různé metody pro zachycení chyb a varování, a to včetně přístupu k jejich následnému ošetření. 

\secc Immediate Error Reporting (Okamžité hlášení chyb)
Tento přístup znamená, že kompilátor zastaví zpracování kódu a vypíše chybu ihned v místě jejího vzniku. Tato metoda je běžná při zachytávání syntaxových chyb. Výhodou okamžitého hlášení je rychlá identifikace chyby, nevýhodou může být časté přerušování procesu překladu, což může být neefektivní, pokud má kód více chyb.

\secc Error Buffering (Ukládání chyb do bufferu)
V tomto přístupu kompilátor shromažďuje chyby v chybovém bufferu (nebo logu), aby mohl detekovat více chyb najednou, než přeruší kompilaci. Tento přístup je užitečný, protože vývojáři poskytuje přehled všech chyb v kódu, což umožňuje efektivnější opravu více problémů najednou. Po dokončení fáze překladu se z bufferu vypíše seznam všech chyb, což zjednodušuje ladění a údržbu kódu.

\secc Recovery Methods (Metody zotavení)
Kompilátory často implementují metody pro zotavení po detekci chyby, což jim umožňuje pokračovat v překladu a najít další chyby. Metody pro zotavení jsou často kombinovány s metodou ukládání chyb do bufferu. Mezi tyto metody patří:
\begitems
* {\bf Panic Mode}: Kompilátor po zjištění chyby ignoruje zbytek aktuálního bloku nebo příkazu a pokračuje až od dalšího vhodného bodu, jako je konec funkce nebo příkaz. Tento přístup je rychlý, ale může vynechat některé chyby v rámci ignorovaného úseku.
* {\bf Phrase-Level Recovery}: Při tomto přístupu se kompilátor pokusí upravit nebo doplnit kód (např. doplněním chybějícího středníku) a pokračovat v překladu, což umožňuje identifikaci dalších chyb. Tento přístup je však složitější a nemusí vždy vést k správnému zpracování kódu.
* {\bf Error Productions}: Jedná se o speciální případ metody Phrase-Level Recovery. Kompilátor je naprogramován tak, aby rozeznal běžné chyby a pokusil se je automaticky opravit. Například, pokud chybí závorka v podmíněném výrazu, kompilátor ji může doplnit, aby umožnil pokračování v překladu. Tato metoda však může být náročná na implementaci a vyžaduje znalost běžných chybových vzorců.
\enditems

\sec Definice chyb a varování v kompilátoru
Definice chyb a varování v kompilátoru závisí na metodě, která jse použita k jejich zachytávání. V případě metody Imediate Error Reporting jsou veškeré informace předány na vstup 







