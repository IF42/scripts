\sec Co je to kompilátor
Kompilátor je počítačový program, který překládá zdrojový kód v určitém jazyce programu v jazyce cílové platformy. Cílová platforma může být buď hardwarový procesor, nebo virtuální stroj. Kompilátory neslouží pouze k přeložení zdrojových kódů programů, ale také pro jejích vylepšení. Kompilátory asistují programátorům při hledání chyb v době kompilace (lexikální, syntaktické a sémantické chyby), ale zároveň umožňují optimalizovat výsledný přeložený program, a odstranit nepotřebné části případně minimalizovat určité sekvence by v paměti zabírali méně místna nebo poskytovali lepší výkon za běhu.

Vedle kompilátorů se pak využívají také interpretery, které se od kompilátorů liší tím, že místo vytvoření spustitelného souboru s přeloženým programem daný program okamžitě vykonávají. Takovému interpreteru se v praxi často říká virtuální stroj. 

Kompilátory a interpretery spolu úzce souvisejí a jsou často zaměňovány. Součástí interpreteru je často kompilátor, který program ve zdrojovém jazyce překládá nejčastěji do podoby {\bf bytecodu}, který umožňuje rychleji vykonávat výsledný program. Jiný způsob je překlad bytecodu do strojového kódu lokálního procesoru a bez uložení do souboru jej vykonáva. To je známé jako {\bf Just-In-Time} kompilace nebo také {\bf JIT}.
