\chap Úvod
Kompilátor je počítačový program, který překládá program napsaný v jednom jazyce na program napsaný v jiném jazyce. To znamená, že kompilátory potřebují znát gramatiku a kontext jak vstupního jazyka tak výstupního jazyka. Struktura kompálátoru pak vychází z tohoto jednoduchého pozorování. Kompilátor jakožto program se skládá z front-end části, která má za úkol zpracovat vstupní jazyk a back-end, který má za úkol pracovat s výstupním jazykem. Front-end a back-end jsou propojeny pomocí vrstvy middle-end která je pracuje s tzv. mezikódem IR.

\vskip 5mm
\picw=.5\hsize \centerline{\inspic {\imgpath compiler_role.png} }\nobreak\medskip

Vedle kompilátorů se pak využívají také interpretery, které se od kompilátorů liší tím, že místo vytvoření spustitelného souboru s přeloženým programem daný program okamžitě vykonávají. Takovému interpreteru se v praxi často říká virtuální stroj. 

Kompilátory a interpretery spolu úzce souvisejí a jsou často zaměňovány. Součástí interpreteru je často kompilátor, který program ve zdrojovém jazyce překládá nejčastěji do podoby {\bf bytecodu}, který umožňuje rychleji vykonávat výsledný program. Jiný způsob je překlad bytecodu do strojového kódu lokálního procesoru a bez uložení do souboru jej vykonáva. To je známé jako {\bf Just-In-Time} kompilace nebo také {\bf JIT}.

Počítačový program je jednoduchá sekvence abstraktních operací zapsaná v {\bf Programovacím jazyce} - formální jazyk, navržený pro vyjádření výpočtů. 
 
