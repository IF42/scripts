
\chap {Modulový systém}
Modulový systém v programovacích jazycích je mechanismus, který slouží k organizaci a strukturování kódu do logických částí zvaných moduly. Modulový systém umožňuje programátorům rozdělit rozsáhlý kód do menších, lépe spravovatelných jednotek a zajišťuje kontrolu nad viditelností a přístupností symbolů (jako jsou funkce, proměnné, třídy) mezi těmito jednotkami. To poskytuje několik klíčových výhod:


Modulový systém určuje, jakým způsobem jsou organizovány a spravovány různé části kódu. To ovlivňuje nejen syntax a sémantiku jazyka, ale i to, jak překladač zpracovává různé části kódu.
