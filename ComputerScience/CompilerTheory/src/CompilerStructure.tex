\chap {Struktura kompilátoru}
Moderní kompilátory jsou navrženy tak, aby byly flexibilní a modulární. Tato flexibilita umožňuje kompilátoru efektivně zpracovávat různé programovací jazyky (na vstupu) a generovat kód pro různé platformy (na výstupu). Klíčem k této modularitě je standardizovaná intermediate representation (IR), kterou využívá střední část architektury kompilátoru - tzv. middle-end. 

Kompilátor se nejčastěji skládá ze tří základních logických částí:
\begitems
* {\bf Front-End} - Zpracovává vstupní programový kód.
* {\bf Middle-End} - mechanismy, které převádí abstraktní syntaktický strom na mezikód IR, případně provádějí některé platformě nezávislé optimalizace. Dále se provádí alokace paměti a registrů a jiných zdrojů procesoru. 
* {\bf Back-End} - definice cílové platformy. Na této úrovni se provádějí platformně závislé optimalizce a převádí se IR na instrukce procesoru
\enditems

Výhodou tohoto členění architektury komilátoru je, že v případě, že je standardizovaný formát IR na vrstvě {\bf Middle-End} je možné dynamicky měnit části {\bf Front-End} pro zpracovávání zcela jiného programovacího jazyka a nebo {\bf Back-End} pro generování strojových instrukcí pro zcela jinou platformu se zachování stejné gramatiky programovacího jazyka. Změna cílového jazyka, který kompilátor generuje na svém výstupu se nazývá {\bf Compiler Retargeting} 


\vskip 5mm
\picw=\hsize \centerline{\inspic {\imgpath compiler_structure.png} }\nobreak\medskip
\caption/f Struktura kompilátoru


\sec{Fáze překladu}
Kompilátor je jednou z komponent nástrojové sady {\bf (toolchain)} použivané pro vytvření spustitelného souboru z programu ve formě textu ve zdrojovém jazyce. Typicky při spuštění jediného příkazu k překladu je vyvolána celá sekvence prorgamů na pozadí. 

\vskip 5mm
\picw=\hsize \centerline{\inspic {\imgpath compiler_process_overview.png} }\nobreak\medskip
\caption/f Proces překladu zdrojových kódů

\secc Preprocesing
Preprocesor je nástroj v procesu překladu zdrojového kódu, jehož účelem je provést úpravy kódu před samotnou kompilací. Preprocesor zpracovává specifické příkazy nebo direktivy, které mohou zahrnovat vkládání externích souborů, definici konstant a maker, podmíněné sestavení kódu nebo jeho úpravy ještě před analýzou syntaxe. Tímto způsobem může vývojář upravovat kód dynamicky podle prostředí, zvyšovat přehlednost pomocí symbolických konstant, či minimalizovat opakování kódu s použitím maker. Výstupem preprocesingu je čistý kód, připravený pro kompilaci, který splňuje požadavky a strukturu definovanou direktivami preprocesoru. Preprocesor samotný ale není nezbytnou součástí kompilátorů, proto nemusí být ve všech případe v kompilátorech přítomné.

\secc{Kompilace}
Fáze kompilace je klíčovým krokem v procesu překladu zdrojového kódu na spustitelný kód. Během této fáze kompilátor analyzuje syntaxi a sémantiku kódu, převádí ho z původního programovacího jazyka do střední reprezentace (IR) a následně aplikuje optimalizace, které zlepší efektivitu výsledného programu. Poté kompilátor IR transformuje na instrukce konkrétní strojové architektury nebo na assemblerový kód, který je blízký instrukční sadě procesoru. Účelem fáze kompilace je tedy vytvořit nízkoúrovňový kód, který je přesně přizpůsoben cílové platformě, přičemž zachovává logiku a funkčnost původního zdrojového kódu.

\secc{Assembler}
Fáze překladu assemblerového kódu do objektového kódu slouží k převodu nízkoúrovňových instrukcí, napsaných v assembleru, na strojový kód, který může být přímo interpretován procesorem. V této fázi assembler převádí každou instrukci na konkrétní binární reprezentaci odpovídající instrukční sadě cílového procesoru. Kromě toho vytváří tabulky symbolů a další metadata, která pomáhají při pozdějším sestavení (linkování). Výstupem této fáze je objektový kód, který není zatím kompletním spustitelným programem, ale obsahuje všechny základní instrukce a data, které budou spojeny a finalizovány ve fázi linkování.

\secc {Linkování}
Fáze linkování je posledním krokem v překladu programu, jehož účelem je spojit jednotlivé objektové soubory a knihovny do jednoho kompletního spustitelného souboru. Linker v této fázi vyhledá a propojí všechny odkazy na symboly (například funkce a globální proměnné) mezi různými částmi programu a knihovnami. Rovněž přiřadí výsledné adresy paměti pro tyto symboly, čímž zajistí, že programové instrukce a data jsou správně propojené a připravené k běhu. Výsledkem je hotový, spustitelný program, který může operační systém načíst a provést.

\sec {Fáze kompilace} 
Fáze kompilace je klíčovou součástí procesu překladu, ve kterém je zdrojový kód napsaný v programovacím jazyce převáděn do podoby, kterou lze efektivně provést na cílové platformě. Tato fáze zahrnuje několik kroků, které postupně transformují zdrojový kód do podoby, která je blízká strojovému kódu, a přitom optimalizuje výkon výsledného programu. V procesu kompilace lze rozlišit několik základních kroků, lexikální analýza, syntaktická analýza, sémantická analýza, generování mezikódu (IR), optimalizace, generování kódu cílové platformny.

\vskip 5mm
\picw=\hsize \centerline{\inspic {\imgpath compile_steps.png} }\nobreak\medskip
\caption/f Proces překladu zdrojových kódů

\secc {Lexikální analýza}
Lexikální analýza je prvním krokem ve fázi kompilace, kde kompilátor rozkládá zdrojový kód na základní jednotky zvané {\bf tokeny}. Tokeny jsou sekvence znaků, které tvoří logické stavební bloky kódu, jako jsou klíčová slova, identifikátory, operátory a literály. Lexikální analyzátor prochází zdrojový kód, identifikuje tyto tokeny a zpracovává je tak, aby byly připraveny pro další fázi. Pokud zjistí chyby, například neznámé znaky, kompilace je přerušena a uživatel je informován.


\secc {Syntaktická analýza}
Během syntaktické analýzy kompilátor vytváří strukturu programu, která odpovídá gramatice daného jazyka. Syntaktický analyzátor používá tokeny generované lexikálním analyzátorem k vytvoření {\bf syntaktického stromu (AST)}, který zobrazuje hierarchickou strukturu kódu. Tento strom odhaluje vztahy mezi jednotlivými konstrukcemi jazyka, jako jsou výrazy, příkazy a bloky kódu. Tato fáze zajišťuje, že program odpovídá pravidlům jazyka a že má správnou syntaxi.

\secc {Sémantická analýza}
V sémantické analýze kompilátor kontroluje, zda syntaktický strom kódu dodržuje významová pravidla jazyka, tedy že je program logicky správný. Během této fáze se provádí například kontrola typů proměnných, deklarací a definic, volání funkcí nebo kompatibility operandů v operacích. Výsledkem je zajištění, že program má platný význam a bude fungovat správně v kontextu daného jazyka. Sémantická analýza také rozšiřuje syntaktický strom o doplňující informace, které jsou později využívány při optimalizaci.

\secc {Optimalizace}
Optimalizace je klíčovou fází, která umožňuje kompilátoru zlepšit výkonnost a efektivitu výsledného kódu. Optimalizační algoritmy se aplikují na střední reprezentaci a zajišťují například minimalizaci počtu instrukcí, efektivní využití paměti nebo redukci redundancí. Existují různé úrovně optimalizací - od základních, jako je eliminace zbytečných výpočtů, až po pokročilé techniky, jako je paralelizace nebo inlining funkcí. Optimalizace se snaží nalézt rovnováhu mezi výkonem programu a jeho velikostí.

\secc {Generování cílvého kódu}
V závěrečné fázi kompilace je optimalizovaná střední reprezentace převedena na cílový kód specifický pro platformu, pro kterou je program určen. Kompilátor převádí IR na instrukce v assembleru, nebo přímo na strojový kód, přičemž zohledňuje konkrétní instrukční sadu procesoru cílové architektury. Výstupem této fáze je sada instrukcí, které jsou interpretovatelné procesorem a zajišťují požadovanou funkčnost programu.




