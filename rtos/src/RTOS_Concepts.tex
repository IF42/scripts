\chap Základní části RTOS


\sec Úlohy (Tasks / Threads)
Úloha (někdy označovaná jako task nebo vlákno) je základní jednotkou běhu programu v RTOSu.

\secc Kód
Funkce, kterou vykonává.

\secc Stav 
Vnitřní stav ve kterém se úloha v systému nachází. RTOS se stará o přechody mezi těmito stavy.

\begitems
* {\bf Running (běží)} - úloha právě využívá CPU.
* {\bf Ready (připravená)} - úloha čeká na spuštění, jakmile jí to plánovač dovolí.
* {\bf Blocked (čeká)} - úloha čeká na splnění podmínky (např. signál, dokončení I/O).
* {\bf Suspended (pozastavená)} - úloha je zastavená až do další aktivace.
\enditems

\secc Priorita
Číselná hodnota, která určuje, jak je úloha důležitá.

\secc Stack
Vlastní zásobník pro uložení kontextu, který se skládá z lokálních proměnných a návratových adres. 

\sec Plánovač (Scheduler)
{\bf Plánovač} je jádro RTOSu. Rozhoduje, která úloha poběží v daný okamžik. Existuje několik hlavních algoritmů:
\begitems
* {\bf Round Robin} - jednoduché střídání úloh v kruhu (každá úloha dostane krátký časový úsek - time slice).
* {\bf Prioritní plánování} - úlohy s vyšší prioritou běží přednostně.
* {\bf Preemptivní plánování} - pokud se objeví úloha s vyšší prioritou, RTOS okamžitě přeruší běžící úlohu a spustí důležitější.
* {\bf Kooperativní plánování} - úloha musí sama předat řízení, RTOS ji nesmí přerušit (jednodušší implementace, ale horší spolehlivost).
\enditems

Většina moderních RTOS používá preemptivní prioritní plánování.

\sec Kontext a jeho přepínání (Context Switching)
Protože více úloh sdílí jedno CPU, musí RTOS při přepnutí úlohy:
\begitems
* uložit stav registrů (obsah CPU registrů, čítač instrukcí, zásobník)
* obnovit stav jiné úlohy
\enditems

Tento proces se nazývá {\bf context switch}. Je to klíčová operace RTOSu a musí být co nejrychlejší, aby systém nebyl zahlcen jen přepínáním.

\sec Synchronizace úloh
Když více úloh spolupracuje nebo sdílí zdroje (např. UART, I2C, paměť), je nutný určitý způsob synchronizace přístupu ke sdíleným zdrojům:
\begitems
* {\bf Semafory} - binární nebo počítané signály, které určují přístup k prostředku.
* {\bf Mutexy (Mutual Exclusion)} - zajišťují, že prostředek používá vždy jen jedna úloha.
* {\bf Fronty (Queues)} - úlohy si posílají zprávy / data prostřednictvím fronty.
* {\bf Události (Events)} - RTOS umožňuje úlohu probudit, jakmile nastane určitá podmínka.
\enditems

\sec Časovač
RTOS má obvykle systémový časovač (tick timer), který:
\begitems
* udává rytmus pro plánovač (např. každých 1 ms)
* umožňuje nastavit zpoždění (delay) nebo periodické úlohy
* umožňuje měřit čas
\enditems


\sec Přerušení (Interrupts)
RTOS spolupracuje s přerušením od hardwaru:
\begitems
* {\bf ISR (Interrupt Service Routine)} - funkce, která se vykoná při přerušení.
* ISR by měla být co nejkratší a pokud je potřeba složitější akce, měla by ISR jen probudit úlohu, která ji vykoná.
\enditems


